\documentclass[11pt]{amsart}

\usepackage{amsthm, amssymb,amsmath}
\usepackage{graphicx}

\theoremstyle{definition}  % Heading is bold, text is roman
\newtheorem{theorem}{Theorem}
\newtheorem{definition}{Definition}
\newtheorem{example}{Example}

\newcommand{\ojo}[1]{{\sffamily\bfseries\boldmath[#1]}}

\oddsidemargin 0pt
\evensidemargin 0pt
\marginparwidth 0pt
\marginparsep 10pt
\topmargin -10pt
\headsep 10pt
\textheight 8.4in
\textwidth 7in

%\input{../header}


\begin{document}

%\homework{}{Homework III}


\noindent For this homework, include all code and computations in a {\tt MATLAB} file named {\tt math425hw3.m}.
You will need to submit this file along with a document containing your answers which do not
involve {\tt MATLAB}.\\


\noindent
{\bf 1.a)} Let ${\bf v} \in \mathbb{R}^m$ and ${\bf w} \in \mathbb{R}^n$. Show that the $m \times n$ matrix ${\bf v} {\bf w}^T$ has rank equal to $1$. \\
{\bf b)} Conversely, show that if $A$ is an $m \times n$ matrix with $\mathrm{rank}(A) = 1$, then $A = {\bf v} {\bf w}^T$ for some ${\bf v} \in \mathbb{R}^m$ and ${\bf w} \in \mathbb{R}^n$. \\

\noindent{\bf 2.} Is $\left( \begin{array}{r} 3 \\ 0 \\ -1 \\ -2 \end{array} \right)$  a linear combination of 
$ \left( \begin{array}{r} 1 \\ 2 \\ 0 \\ 1 \end{array} \right), \quad   \left( \begin{array}{r} 0 \\ -1 \\ 3 \\ 0 \end{array} \right), \quad
\left( \begin{array}{r} 2 \\ 0 \\ 1 \\ -1\end{array} \right)$? Give an answer using {\tt MATLAB}. \\

\noindent {\bf 3.} Let 
$$  {\bf v}_1 = \left( \begin{array}{r} 1 \\ 0 \\ 2  \end{array} \right), \,  {\bf v}_2 = \left( \begin{array}{r} 3 \\ -1 \\ 1  \end{array} \right), \,
{\bf v}_3 = \left( \begin{array}{r} 2 \\ -1 \\ -1  \end{array} \right), \,    {\bf v}_4 = \left( \begin{array}{r} 4 \\ -1\\ 3  \end{array} \right).$$
Use {\tt MATLAB}, if convenient, to answer the following questions. \\
{\bf a)}  Do ${\bf v}_1, {\bf v}_2, {\bf v}_3, {\bf v}_4$ span $\mathbb{R}^3$? Why or why not? \\ 
{\bf b)}  Are ${\bf v}_1, {\bf v}_2, {\bf v}_3, {\bf v}_4$ linearly independent ? Why or why not? \\ 
{\bf c)}  Do ${\bf v}_1, {\bf v}_2, {\bf v}_3, {\bf v}_4$ form a basis for  $\mathbb{R}^3$? Why or why not? If not, is it possible to choose some
subset which is a basis? \\ 
{\bf d)}  What is the dimension of the span of ${\bf v}_1, {\bf v}_2, {\bf v}_3, {\bf v}_4$ ? Justify your answer. \\


\noindent {\bf 4.a)}  Create a function called {\tt myGS} which takes as input an $m \times n$ matrix $A$ where $\mathrm{rank}(A) = n \leq m$.  The output is an $m \times n$
matrix $B$ whose columns form an orthonormal basis of the vector space spanned by the columns of $A$. Use the Gram-Schmidt process. \\
{\bf b)} Use {\tt myGS} to compute an orthonormal basis for $\mathbb{R}^4$ starting with the following set of vectors:
$$ \left( \begin{array}{r} 1 \\ 0   \\  1 \\ 0\end{array} \right), \quad
\left( \begin{array}{r} 0\\ 1 \\ 0 \\  -1 \end{array} \right), \quad
\left( \begin{array}{r}  1 \\ 0 \\ 0 \\ 1 \end{array} \right), \quad \left( \begin{array}{r} 1\\ 1 \\ 1 \\  1\end{array} \right).$$\\
{\bf c)}  Modify your function to {\tt myGS2} so that it computes an orthonormal basis ``on the fly'' (as we have learned last week). Use {\tt myGS2} on the input in part {\bf b)}. \\




 


\end{document}


