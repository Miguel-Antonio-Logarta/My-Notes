\documentclass{article}

\usepackage{fancyhdr}
\usepackage{amsmath}

\title{CSC MATH 425 Homework 2}
\date{Due: Saturday Sep 21, 2024}
\author{Miguel Antonio Logarta}

\pagestyle{fancy}

\fancyhead[L]{SFSU, MATH 425}  % Left side of the header
\fancyhead[C]{Fall 2024}  % Center of the header
\fancyhead[R]{Due: Saturday Sep 21, 2024}  % Right side of the header (current date)

\begin{document}

\maketitle  % This command generates the title page.

\begin{itemize}
\item[1.c] Yes, my function \textit{myRank()} computes the rank of \textit{A} to be 3.
\item[2.a] An example of a $4 \times 4$ strictly \underline{column} diagonally dominant matrix is:$$\begin{bmatrix} 3&1&1&0 \\ 2&4&1&0 \\ 0&1&5&2 \\ 1&1&3&6\end{bmatrix}$$In this matrix, the magnitude of every diagonal entry is greater than the sum of every non-diagonal entry in their column.
\item[2.b]Using this inequality, $${\displaystyle |a_{jj}|> \sum ^{n}_{i\neq j}|a_{ij}|}$$ We can show that Gaussian elimination with partial pivoting on the array from 2.a will have zero row interchanges. When we use partial pivoting, the algorithm looks for values below $a_{jj}$ and finds that there are no rows whose value is greater than the current pivot. No row interchanges occur, and algorithm moves on to the next diagonal.

\begin{itemize}
    \item[1.] In the first column we see that:\newline $a_{11} > a_{12}$, $a_{11} > a_{13}$, $a_{11} > a_{14}$. \newline No row interchanges occur.
    \item[2.] In the second column we see that:\newline $a_{21} > a_{22}$, $a_{21} > a_{23}$, $a_{21} > a_{24}$. \newline No row interchanges occur. 
    \item[3.] Third column:\newline $a_{31} > a_{32}$, $a_{31} > a_{33}$, $a_{31} > a_{34}$. \newline No row interchanges occur.
    \item[4.] Fourth column:\newline $a_{41} > a_{42}$, $a_{41} > a_{43}$, $a_{41} > a_{44}$. \newline No row interchanges occur.
\end{itemize}

\item[2.c] This example is used in Matlab and the results indeed show that zero row interchanges have occured. 
\end{itemize}

\end{document}
