\documentclass[11pt]{amsart}

\usepackage{amsthm, amssymb,amsmath}
\usepackage{graphicx}

\theoremstyle{definition}  % Heading is bold, text is roman
\newtheorem{theorem}{Theorem}
\newtheorem{definition}{Definition}
\newtheorem{example}{Example}

\newcommand{\ojo}[1]{{\sffamily\bfseries\boldmath[#1]}}

\oddsidemargin 0pt
\evensidemargin 0pt
\marginparwidth 0pt
\marginparsep 10pt
\topmargin -10pt
\headsep 10pt
\textheight 8.4in
\textwidth 7in

%\input{../header}


\begin{document}

%\homework{}{Homework VII}


\noindent For this homework, include all code and computations in a {\tt MATLAB} file named {\tt math425hw7.m}.
You will need to submit this file along with a document containing your answers which do not
involve {\tt MATLAB}. {\bf Do not submit a zipped (compressed) folder}. \\

\noindent
{\bf 1.} Let $A$ be a nonsingular $n \times n$ matrix with real entries and $b \in \mathbb{R}^n$. Explain carefully how you can use the SVD of $A$ to solve the system
of linear equations $Ax=b$. Why is $A$ being nonsingular important ? \\

\noindent
{\bf 2.}  Let $A$ be a nonsingular $n \times n$ matrix with real entries. How are the singular values of $A$ and the singular values of $A^{-1}$  related? Justify.\\


\noindent
{\bf 3.}  Let $A$ be an $m \times n$ matrix with real entries $a_{ij}$. We will denote $\sqrt{\sum_{j=1}^n \sum_{i=1}^m a_{ij}^2}$ by $||A||$. \\
{\bf a)} Let $B$ be a $p \times r$ matrix and $C$ be a $r \times p$ matrix. Prove that $\mathrm{trace}(BC) = \mathrm{trace}(CB)$. \\
{\bf b)} Show that $||A||^2 = \mathrm{trace}(AA^T) = \mathrm{trace}(A^TA)$. \\
{\bf c)} Let $U$ be an $m \times m$ orthogonal matrix. Prove that $||UA|| = ||A||$. \\
{\bf d)} Now let $\sigma_1 \geq \sigma_2 \geq \cdots \geq \sigma_r >0$ be the singular values of $A$. Show that $||A|| = \sqrt{\sigma_1^2 + \sigma_2^2 + \cdots + \sigma_r^2}$. \\

\noindent
{\bf 4.a)} Let $A$ be an $m \times n$ matrix with real entries and let $A = P\Sigma Q^T$ be its singular value decomposition. Let $p_1, p_2, \ldots, p_r$ be the columns of $P$
and let $q_1, q_2, \ldots, q_r$ be the columns of $Q$. Show that $ A = \sigma_1 p_1 q_1^T + \cdots + \sigma_r p_r q_r^T$. \\
{\bf 4.b)} Now let $A_k = P_k \Sigma_k Q_k^T$ be the truncated SVD as we have done in the class. Show that $||A - A_k|| = \sqrt{\sigma_{k+1} + \cdots + \sigma_{r}}$. [Hint: obtain
the SVD of $A - A_k$ by using {\bf 4.a)}, then use {\bf 3.d)}] \\

\noindent
{\bf 5.a)} Upload an image into your {\tt MATLAB} directory. Using {\tt imread} and {\tt im2gray} (if necessary) and {\tt im2double} store the image in a matrix $A$.\\
{\bf 5.b)} Compute the SVD of $A$ using {\tt MATLAB}, and using various truncated matrices $A_k$ of rank $k$ determine a small $k$ for which the image generated from $A_k$
is a good approximation of the image generated by from $A$. [{\tt imshow} displays the image] \\
{\bf 5.c)} Pay attention to the singular values of $A$. By looking at them could you have predicted a good value of $k$~? Elaborate. \\





 


\end{document}


